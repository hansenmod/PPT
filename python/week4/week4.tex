\documentclass{beamer}
\usepackage{ctex}
\usepackage{zhs}
\usepackage{diagbox}  % 斜线表头
\usepackage{colortbl} %
\usepackage{minted}
\title{python语言程序设计基础}
\author{Hengsheng Zhou}
\institute{电信与智能制造学院}
\begin{document}
\begin{frame}[t]
	\titlepage
	\begin{figure}
		\begin{center}
			\includegraphics[width=0.2\linewidth]{output.eps}
		\end{center}
	\end{figure}


\end{frame}
\begin{frame}[h]
\frametitle{回顾}
请同们用先思考一下列表的特性

\end{frame}
\begin{frame}[fragile]
	\frametitle{元组}
\framesubtitle{元组的创建}
\begin{table}[htpb]
\centering
\label{tab:label}
\begin{tabular}{|c|c|}
\hline
列表 & 元组 \\
\hline
有序   & 有序    \\
\hline
有重复元素  & 有重复元素  \\
\hline
元素值可更改 & 元素值可更改 \\
\hline
可增删元素  & 不可增删元素 \\
\hline
\end{tabular}
\end{table}
\pause
\begin{columns}
	\begin{column}{0.5\textwidth}
\begin{minted}[fontsize=\small, linenos]{python}
mytuple1=(1,2,3,4,5)
#使用中括号创建元组
\end{minted}
	\end{column}
	\begin{column}{0.5\textwidth}
\begin{minted}[fontsize=\small, linenos]{python}
mytuple2=tuple({1,2,3,4})
mytuple2=tuple((1,2,3,4))
mytuple2=tuple({1,2,3,4})
#使用构造器创建元组
\end{minted}	
	\end{column}
\end{columns}

\pause
\begin{alertblock}{定义但个元素的元组}
\begin{minted}[fontsize=\small, linenos]{python}
thistuple = ("apple",)
#将创建一个元组
thistuple = ("apple")
#将创建一个字符
\end{minted}
\end{alertblock}
\end{frame}


\begin{frame}[t]
	\frametitle{练习}
	\begin{example}[1]

		使用构造器创建一个具有10000个随机元素的元组
	\end{example}

\end{frame}
\begin{frame}
	\frametitle{元组}
	\framesubtitle{访问元组}
	\begin{itemize}
		\item 索引(正向,反向,截取)
		\item 遍历
		\item 筛选
	\end{itemize}
	\pause
	\begin{example}[1]
		访问元组的所有索引值为偶数的元素
	\end{example}
\end{frame}

\begin{frame}
	\frametitle{元组}
	\framesubtitle{添加元素/删除元素}
	因为tuple是不可更改的,如果需要更改tuple中的元素需要将其转化为list类型的变量

\end{frame}

\begin{frame}
	\frametitle{元组}
	\framesubtitle{解包}
	将元组中的元素一次赋给多个变量
	\begin{block}{example}
		fruits = ("apple", "banana", "cherry", "strawberry", "raspberry")

		(green, yellow, *red) = fruits

		print(green)
		print(yellow)
		print(red)
	\end{block}
	\pause
	\begin{example}[1]
		定义一个函数使用解包的方式为函数赋值
	\end{example}
\end{frame}

\begin{frame}[t]
	\frametitle{元组}
	\framesubtitle{方法}
	\begin{itemize}
		\item count() :输出某个元素在tuple中出的次数
		\item index() :输出某个元素在元组中第一次出现位置的索引值
	\end{itemize}

\end{frame}
\begin{frame}[t]
	\frametitle{集合set}
	\framesubtitle{创建集合}
	\begin{table}[htpb]
		\centering
		\label{tab:label}
		\begin{tabular}{|c|c|c|}
			\hline
			列表     & 元组     & 集合     \\
			\hline
			有序     & 有序     & 无序     \\
			\hline
			有重复元素  & 有重复元素  & 无重复值   \\
			\hline
			元素值可更改 & 元素值可更改 & 元素值不可变 \\
			\hline
			可增删元素  & 不可增删元素 & 可增删元素  \\
			\hline
		\end{tabular}
	\end{table}
	\pause
	\begin{example}
		{1,2,3}\\
		set(list or set)
	\end{example}

\end{frame}
\begin{frame}[t]
	\frametitle{讨论}
	请同学们讨论一下1和True或0和False可以在同一个集合中么?

	
\end{frame}
\begin{frame}[t]
	\frametitle{集合}
	\framesubtitle{访问set}
	\begin{itemize}
		\item 索引
		\item 遍历
		\item 筛选
		\item 查存

		      \begin{block}{查存}
			      thisset = {"apple", "banana", "cherry"}\\
			      print("banana" in thisset)
		      \end{block}

	\end{itemize}

\end{frame}

\begin{frame}[t]
	\frametitle{集合}
	\framesubtitle{向集合中添加元素}
	\begin{itemize}
		\item add()
		\item update()更新原集合、union()=|返回新集合
	\end{itemize}
\end{frame}

\begin{frame}[t]
	\frametitle{集合}
	\framesubtitle{删除集合中的元素}
	\begin{itemize}
		\item remove()
		\item discard()
		\item pop()
		\item clear()
	\end{itemize}
\end{frame}
\begin{frame}[t]
	\frametitle{集合}
	\framesubtitle{集合中的方法}
	\begin{itemize}
		\item intersection() 	取两个集合的交集
		\item intersection\_update()在原集合上更改
		\item difference()	等同于A-B将在另一个集合中出现过的元素筛掉形成新集合
		\item difference\_update()在元集合上更改
		\item symmetric\_difference() 	等同于(A-B)+(B-A)
		\item symmetric\_difference\_update()在原集合上更改

	\end{itemize}


\end{frame}
\begin{frame}[t]
	\frametitle{字典}
	\framesubtitle{创建字典}
	\begin{table}[htpb]
		\centering
		\label{tab:label}
		\begin{tabular}{|c|c|c|c|}
			\hline
			列表     & 元组     & 集合     & 字典    \\
			\hline
			有序     & 有序     & 无序     & 有序    \\
			\hline
			有重复元素  & 有重复元素  & 无重复值   & 无重复值  \\
			\hline
			元素值可更改 & 元素值可更改 & 元素值不可变 & 元素值可变 \\
			\hline
			可增删元素  & 不可增删元素 & 可增删元素  & 可增删元素 \\
			\hline
		\end{tabular}
	\end{table}
	\pause

	字典是有序、可改值、不允许重复的集合\\
	\{1,2,3,4\}\\
	dict(list or tuple or set)\\
	{}.fromkeys('abc')
	\pause
	\begin{example}[1]
		生成一个具有1000个元素的字典,其中元素的值分别为0-100的随机值
	\end{example}
\end{frame}

\begin{frame}[t]
	\frametitle{字典}
	\framesubtitle{字典元素的访问}
	\begin{itemize}
		\item get():通过key值访问某个元素的value
		\item keys():返回所有的keys
		\item values():返回所有的values
		\item items():返回所有的items
		\item 遍历字典
		      \begin{itemize}
			      \item for x in dictionary 和keys():遍历keys
			      \item for x in dictionary: dictionary[x] 和values()便利values
			      \item for x,y in dictionary.items: 遍历(key,value)
		      \end{itemize}

	\end{itemize}


\end{frame}
\begin{frame}[t]
	\frametitle{讨论}
	请同学们讨论一下若使用keys获取到所有键位后更改了字典那么以获取的键位列表会不会改变?

	
\end{frame}
\begin{frame}[t]
	\frametitle{字典}
	\framesubtitle{向字典中添加元素}
	\begin{itemize}
		\item dictionary[keys]=values
		\item update(): 函数的值可以是任何iterable类型的变量
	\end{itemize}

\end{frame}
\begin{frame}[t]
	\frametitle{字典}
	\framesubtitle{删除字典中的元素}
	\begin{itemize}
		\item pop(key)删除指定key的元素
		\item popitem()删除最后一个元素
		\item clear()
	\end{itemize}
\end{frame}

\end{document}
