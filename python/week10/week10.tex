\documentclass{beamer}
\usepackage{ctex}
\usepackage{zhs}
\usepackage{diagbox}  % 斜线表头
\usepackage{tabularx}
\usepackage{array}    % 导言区
\usepackage{makecell} % 可选,增强表头美观
\usepackage{minted}
\title{python语言程序设计基础}
\author{Hengsheng Zhou}
\institute{电信与智能制造学院}
\begin{document}
\begin{frame}[t]
	\titlepage
	\begin{figure}
		\begin{center}
			\includegraphics[width=0.2\linewidth]{output.eps}
		\end{center}
	\end{figure}


\end{frame}
\begin{frame}
	\frametitle{Outline}
	\tableofcontents
\end{frame}
\section{Class}%
\subsection{定义}%
\begin{frame}[t]
	\frametitle{Class}
	\framesubtitle{定义|创建}
	类将一些字段和方法归纳在一起具有面向对象的特性。
	\pause
	\begin{center}
		\begin{tabular}{|c|c|}
			\hline
			继承 & 子类继承父类的方法和属性 \\
			\hline
			多态 & 同一接口不同对象行为不同 \\
			\hline
		\end{tabular}
	\end{center}
	\pause
	用class关键字创建一个类
	\begin{example}
		class MyClass():


	\end{example}
\end{frame}

\subsection{初始化}%
\begin{frame}[t]
	\frametitle{Class}
	\framesubtitle{初始化}
	def \_\_init\_\_(self):是类的实例化方法
	当某个类的对象被创建时\_\_init\_\_初始化方法总是先被执行
	\pause
	\begin{itemize}
		\item 实例变量
		\item 受保护的变量
		\item 私有变量
		\item 类变量
	\end{itemize}
	\pause
	\begin{example}
		写一个初始化函数,该函数为对象的三个字段赋值
	\end{example}


\end{frame}

\subsection{魔术方法}%
\begin{frame}[fragile]
	\frametitle{Class}
	\framesubtitle{魔术方法}
	重写魔术方法可以改变类的某些特定行为,以两个下划线开头和两个下划线结尾的函数都是python里的魔术方法
	\pause
	\begin{example}
		\begin{itemize}
			\item \_\_init\_\_()
			\item \_\_str\_\_()
			\item \_\_next\_\_()
			\item \_\_iter\_\_()
			\item \_\_eq\_\_()
		\end{itemize}

	\end{example}

\end{frame}

\subsection{实例方法|类方法|静态方法|}%
\begin{frame}[fragile]
	\frametitle{Class}
	\framesubtitle{实例方法}
	\begin{table}[htpb]
		\centering
		\begin{tabular}{| >{\centering\arraybackslash}m{3cm} | >{\centering\arraybackslash}m{3cm} | >{\centering\arraybackslash}m{3cm} |}			\hline
			实例方法 & 参数中带有self形参的方法被称为实例方法 & 通过对象访问    \\
			\hline
			类方法  & 参数中带有cls形参的方法被称为实例方法  & 通过对象或类名访问 \\
			\hline
			静态方法 & 参数中不具有cls或self的形参     & 通过对象或类名访问 \\
			\hline
		\end{tabular}	\end{table}
\end{frame}


\section{字段}%

\subsection{局部|全局}%

\begin{frame}[fragile]
\frametitle{字段}
\framesubtitle{局部|全局}
\begin{columns}
\begin{column}{0.5\textwidth}
  \begin{minted}[fontsize=\small, linenos]{python}
x=0
def add_self(name):
  x=0
  for _ in range(10):
    x+=1
print(x)
  \end{minted}
\end{column}
\begin{column}{0.5\textwidth}
  \begin{minted}[fontsize=\small, linenos]{python}
x=0
def add_self(name):
  global x
  x=0
  for _ in range(10):
    x+=1
print(x)
  \end{minted}
\end{column}
\end{columns}
\end{frame}

\subsection{私有|受保护的变量|类|实例}%

\begin{frame}[t]
	\frametitle{字段}
	\framesubtitle{私有|受保护的变量|类|实例}
	\begin{center}
		\begin{tabular}{|c|c|}
			\hline
			私有 & 前缀为\_\_ \\
			\hline
			受保护的变量 & 前缀为\_ \\
			\hline
			类变量 & 通过cls赋值 \\
			\hline
			实例变量 & 通过self赋值 \\
			\hline
		\end{tabular}
	\end{center}
\end{frame}




\end{document}
