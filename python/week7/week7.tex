
\documentclass{beamer}
\usepackage{ctex}
\usepackage{zhs}
\usepackage{diagbox}  % 斜线表头
\usepackage{colortbl} %
\title{python语言程序设计基础}
\author{Hengsheng Zhou}
\institute{电信与智能制造学院}
\begin{document}
\begin{frame}[t]
	\titlepage
	\begin{figure}
		\begin{center}
			\includegraphics[width=0.2\linewidth]{output.eps}
		\end{center}
	\end{figure}


\end{frame}
\begin{frame}
	\frametitle{Outline}
	\tableofcontents
\end{frame}

\section{函数}

\subsection{形参不可变函数的定义}

\begin{frame}[t]
	\begin{block}{definition}
		函数是一个实现特定功能的代码段

	\end{block}
	\pause
	\begin{block}{example}
		无参函数
		def my\_function():\\
		print("Hello from a function")
		带有参数的函数
		def my\_function(name):\\
		print(name + " say hello")
		带有两个参数的函数
		def my\_function(name, something):\\
		print(name + "say" + something)

	\end{block}
	\pause

	\begin{alertblock}{Attention}
		调用函数时传入的参数数量必须与形参个数相同。
	\end{alertblock}
	\pause
	\begin{alertblock}{Attention}
		如果在定义函数时无法确定函数形参数量该如何定义函数?
	\end{alertblock}



\end{frame}
\subsection{可变形参函数的定义}

\begin{frame}[t]
	\begin{block}{创建可变形参函数}
		def my\_function(*kids):\\
		print("The youngest child is " + kids[2])\\
	\end{block}
	\pause
	\begin{block}{调用}
		my\_function("Emil", "Tobias", "Linus")\#将实参作为元组传入函数因此需要通过索引值的方式访问实参
	\end{block}

\end{frame}
\begin{frame}[t]
	\begin{block}{定义}
		def my\_function(**kid):\\
		print("His last name is " + kid["lname"])\\
	\end{block}
	\pause
	\begin{block}{调用}
		my\_function(fname = "Tobias", lname = "Refsnes")\#由于实参以字典的方式传入函数,因此条用时需要指定key
	\end{block}
\end{frame}

\section{函数的调用}

\begin{frame}[t]
	\begin{block}{规定实参形式}
		def test\_function (a,b,/,*,c,d,)\#规定,/符合之前实参按顺序赋值,在,*之后形参按key复制
	\end{block}
	\pause
	\begin{block}{为形参设置默认值}
		def my\_function(country = "Norway")\#如果调用函数不传入实参怎使用默认值

	\end{block}
	\pause
	\begin{block}{空体函数}
		def myfunction():\#通过pass关键字可以暂时不指定函数体\\
		pass
	\end{block}

\end{frame}

\section{Lambda表达式}

\begin{frame}[t]
	\begin{block}{创建}
		x = lambda a, b, c : a + b + c
	\end{block}
	\pause
	\begin{block}{调用}
		def lambda\_func(a,b,*,func):\\定义
		lambda\_func(2,4,func=lambda a,b:a*b)\\调用将lambda表达式作为实参

	\end{block}
\end{frame}

\section{正则表达式}


\begin{frame}[t]
	\begin{columns}

		\begin{column}{0.5\textwidth}
			\begin{table}[htpb]
				\centering
				\caption{基本表达式}
				\label{tab:label}
				\begin{tabular}{|c|c|}
					\hline
					\[\] & 一个字符序列                 \\
					\hline
					位置   & $\^$、\$                \\
					\hline
					数量   & *、+、?、{}               \\
					\hline
					元素   & []、$.$、 \textbackslash \\
					\hline
					逻辑   & |                      \\
					\hline
				\end{tabular}
			\end{table}
		\end{column}
		\pause
		\begin{column}{0.5\textwidth}
			\begin{table}[htpb]
				\centering
				\caption{元素表达式[   ]}
				\label{tab:label}

				\begin{tabular}{|c|c|}
					\hline
					$[abc]$    & abc任意一个   \\
					\hline
					$[a-c]$    & 字母表任意一个   \\
					\hline
					[ $\^$abc] & 除了abc任意一个 \\
					\hline
				\end{tabular}
			\end{table}

		\end{column}

	\end{columns}
	\pause
	\begin{table}[htpb]
		\centering
		\caption{元素表达式\textbackslash}
		\label{tab:label}

		\begin{tabular}{|c|c|}
			\hline
			\textbackslash d & 数字\\
			\hline
			\textbackslash D & 非数字\\
			\hline
			\textbackslash s & 空格\\
			\hline
			\textbackslash S & 非空格\\
			\hline
		\end{tabular}
	\end{table}
\end{frame}



\end{document}
