\documentclass{beamer}
\usepackage{ctex}
\usepackage{zhs}
\usepackage{diagbox}  % 斜线表头
\usepackage{colortbl} %
\title{python语言程序设计基础}
\author{Hengsheng Zhou}
\institute{电信与智能制造学院}
\begin{document}
\begin{frame}[t]
	\titlepage
	\begin{figure}
		\begin{center}
			\includegraphics[width=0.2\linewidth]{output.eps}
		\end{center}
	\end{figure}


\end{frame}
\begin{frame}
	\frametitle{Outline}
	\tableofcontents
\end{frame}

\section{从文件中读取数据}

\subsection{打开文件}

\begin{frame}[t]
	\begin{block}{definition}
		若要读取一个文件的内容需先将文件打开,共有四种打开模式
		\begin{itemize}
			\item r只读模式,若果没有文件则会报错
			      \pause

			\item a和w都是写,若没有指定文件会自动创建,a是在原有基础上添加,w是覆盖
			      \pause

			\item x创建文件,如果指定文件已存在则报错

		\end{itemize}

	\end{block}
	\pause
	\begin{block}{definition}
		文件格式
		\begin{itemize}
			\item t将文件当作文本文件处理
			      \pause

			\item b将文件当作二进制文件处理

		\end{itemize}

	\end{block}
\end{frame}
\subsection{读取文件内的内容}
\begin{frame}[t]
	f=open("filename",rt)\\
	\begin{example}[]
		f.read()\#获取文件内的所有内容
	\end{example}
	\pause
	\begin{example}[]
		f.read(argue)\#获取前argue个字符
	\end{example}
	\pause
	\begin{example}[]
		f.readline()\#每次调用该函数获取下一行字符
	\end{example}

\end{frame}
\subsection{向文件内写入文本}
\begin{frame}[t]
	对比一下两种打开文件方式的区别,理解资源管理的重要性
	\begin{columns}
		\begin{column}{0.5\textwidth}
			f = open("file.txt", "w", encoding="utf-8")\\
			f.write("Hello")\\
			f.close()
		\end{column}
		\pause
		\begin{column}{0.5\textwidth}
			with open("file.txt", "w", encoding="utf-8") as f:\\
			f.write("Hello")
		\end{column}
	\end{columns}

\end{frame}
\subsection{删除文件与目录}
\begin{itemize}
	\item 删除文件
	      \begin{example}[]
		      os.remove("demofile.txt")
	      \end{example}
	\item 删除目录
	      \begin{example}[]
		      os.rmdir("myfolder")
	      \end{example}
\end{itemize}

\section{异常处理}

\begin{frame}[t]
	\frametitle{异常处理}
	try:\\\#执行可能会出错的代码块\\
	expect NameError\\\#前面代码执行时抛出某个具体异常\\
	expect:\\\#抛出除了NameError之外的其他异常\\
	else:\\\#若不抛出异常执行\\
	finally:\\\#无论是否抛出异常均执行


\end{frame}
\end{document}
